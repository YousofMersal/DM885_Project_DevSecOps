\documentclass[a4paper,12pt]{article}
\usepackage[a4paper,margin=2.4cm]{geometry} \linespread{1.25} 
\usepackage[english]{babel}

\usepackage{amsmath}
\usepackage{amssymb}
\usepackage[page,toc]{appendix}
\usepackage{fancyhdr}
\usepackage{fancyvrb}
\usepackage[T1]{fontenc}	
\usepackage{graphicx}
\usepackage[utf8]{inputenc}
\usepackage{titlesec}
\usepackage{hyperref}	

\DeclareGraphicsExtensions{.png}	\graphicspath{{img/}}               
\renewcommand{\arraystretch}{1.15}

\hypersetup{
    colorlinks=true,
    citecolor=gray,
    linkcolor=darkgray,
    urlcolor=violet,
}

\fancyhf{}
\setlength{\headheight}{15pt}
\lfoot{}
\cfoot{}
\rfoot{\thepage{}/\pageref{lastpage}}
\usepackage[newfloat]{minted}
\usepackage{caption}
\captionsetup[listing]{position=bottom,skip=-5pt}

\pagestyle{fancy}
\lhead{Group X}
\chead{Title}
\rhead{\today}

\begin{document}

\begin{titlepage}

\newcommand{\HRule}{\rule{\linewidth}{0.5mm}}
% Defines a new command for 
% horizontal lines, change thickness here

\center % Centre everything on the page

	%------------------------------------------------
	%	Headings
	%------------------------------------------------

	\textsc{\LARGE University of Southern Denmark}\\[1.5cm]
	% Main heading such as the name of your university/college

	\textsc{\Large Department of Mathematics and Computer Science}\\[0.5cm]
	%Major heading such as course name

	\textsc{\large DM885: Microservices and Dev(Sec)Ops}\\[0.5cm]
	% COURSE ID AND NAME

	%------------------------------------------------
	%	Title
	%------------------------------------------------

	\HRule\\[0.7cm]

	{\huge\bfseries Title}\\[0.4cm] % Title of your document

	\HRule\\[1.0cm]

	%------------------------------------------------
	%	Author(s)
	%------------------------------------------------
	\large
	\textit{Authors}\\ \bigskip
	\begin{minipage}[t]{0.45\textwidth}
	    \begin{center}
		    \normalsize
		    X X\\
		    \href{mailto:x@x.x}{x@x.x}\\ \bigskip
		    Z Z\\
		    \href{mailto:x@x.x}{x@x.x}\\ \bigskip

	   \end{center}
	\end{minipage}
	~
	\begin{minipage}[t]{0.45\textwidth}
	    \begin{center}
		    \normalsize
		    Y Y\\
		    \href{mailto:y@y.y}{y@y.y}\\ \bigskip
		\end{center}
	\end{minipage}
	

	%------------------------------------------------
	%	Date
	%------------------------------------------------

	\vfill\vfill\vfill % Position the date 3/4 down the remaining page

	{\large \today}

	%------------------------------------------------
	%	Logo
	%------------------------------------------------

	\vfill\vfill
	\begin{centering}
	\includegraphics[width=0.3\textwidth]{SDU_logo}\\[1cm]
	% Include a 
% department/university logo - this will require the graphicx package
	\end{centering}
	
%-------------------------------------------------------------------------------

	\vfill % Push the date up 1/4 of the remaining page

\end{titlepage}

\pagenumbering{arabic}

\section{Introduction}

The structure is a guideline for the project description. Feel free to insert 
subsections as you
see fit.

This introduction gives an overview of the motivations and results of what you 
have
done. Describe the challenge that you want to address, motivating why it is
relevant.


\section{Preliminaries}

Give a brief overview of the background knowledge needed to understand your
report. Provide references to what you have used. Note: thre is no need to
repeat what we discuss during the course, you should be brief.

\section{Technical Description}

Here you explain the technical work you have carried out. You should present the
overall view of your microservice architecture. You may include code
snippets where relevant, and refer to source code in the project files. Try to
explain the development choices you did taking into account important properties
such as deployability, availability, reusability, security, modifiability,
performance.

\subsection{Architecture}
  
In this subsection please provide at least a diagram of the entire 
architecture and describe it.

\subsection{Infrastructure}

\subsubsection{Automation}

\subsubsection{Security}
    
\subsubsection{Monitoring and Logging} 
    
\subsection{Testing \& Evaluation}

\section{Related Work and Discussion}

In this section you review the relevant state of the art. This may include
alternative solutions to the same challenge you have tried to address in your
project, or alternative methodologies that you may have followed (e.g., choice
of other technologies for implementing the project). Provide a discussion on the
implications of your choices in the design of your work and the
technologies/techniques that you have used.

Discuss what are the shortcomings/limitations of
your project, possibly explaining how they could be solved or mitigated.
    
\section{References}

This section should contain references to the articles/websites/resources/etc.
cited in your report.


    \label{lastpage} % Allows using the page number of this page as the last in 
% the footer
    \newpage
    \pagenumbering{gobble} % Stop numbering pages
    \rfoot{} % Remove page number from footer

% \bibliographystyle{plain}
% \bibliography{xxx.bib}
    

% \appendix
% \section{Whatever you want to add}

\end{document}
